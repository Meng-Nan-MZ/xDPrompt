\documentclass[fontset=windows]{article}

% =====================================================
% 页面设置
% =====================================================
\usepackage[margin=1in]{geometry}
\usepackage{ctex}
\usepackage{float}
\usepackage{setspace}
\usepackage{lipsum}

% =====================================================
% 图片和路径设置
% =====================================================
\usepackage{graphicx}
\graphicspath{{Figures/}}

% =====================================================
% 列表和超链接
% =====================================================
\usepackage{enumitem}
\usepackage{hyperref}
\hypersetup{
    colorlinks = true,
    linkcolor = blue,
    urlcolor = blue,
    citecolor = blue,
    bookmarksopen = true,
    bookmarksnumbered = true,
    pdftitle = {LaTeX标准化模板},
    pdfauthor = {Ali-loner}
}

% =====================================================
% 参考文献
% =====================================================
\bibliographystyle{plain}
\newcommand{\upcite}[1]{\textsuperscript{\cite{#1}}}

% =====================================================
% 目录美化
% =====================================================
\renewcommand{\contentsname}{\centerline{Contents}}
\usepackage{tocloft}
\renewcommand{\cftsecleader}{\cftdotfill{\cftdotsep}}
\setlength{\cftbeforesecskip}{2pt}
\setlength{\cftbeforesubsecskip}{1pt}

% =====================================================
% 页眉页脚设置(可选)
% =====================================================
% \usepackage{fancyhdr}
% \pagestyle{fancy}
% \fancyhf{}
% \fancyhead[L]{文档标题}
% \fancyhead[R]{\leftmark}
% \fancyfoot[C]{\thepage}

% =====================================================
% 章节标题美化
% =====================================================
\usepackage{titlesec}
\titleformat{\section}
  {\normalfont\Large\bfseries\color{blue}}{\thesection}{1em}{}
\titleformat{\subsection}
  {\normalfont\large\bfseries\color{black}}{\thesubsection}{1em}{}
\titleformat{\subsubsection}
  {\normalfont\normalsize\bfseries\color{black}}{\thesubsubsection}{1em}{}

% =====================================================
% 表格美化
% =====================================================
\usepackage{booktabs}
\usepackage{array}
\renewcommand{\arraystretch}{1.2}
\setlength{\tabcolsep}{8pt}

% =====================================================
% 项目符号美化
% =====================================================
\renewcommand{\labelitemi}{\textbullet}

% =====================================================
% 颜色设置
% =====================================================
\usepackage{xcolor}
\definecolor{myblue}{RGB}{0,76,153}
\hypersetup{linkcolor=myblue, urlcolor=myblue, citecolor=myblue}

% =====================================================
% 段落设置
% =====================================================
\setlength{\parskip}{0.5em}
\setlength{\parindent}{2em}

% =====================================================
% 标题页美化
% =====================================================
\makeatletter
\renewcommand{\maketitle}{
  \begin{center}
    {\heiti\zihao{2} \@title \par}
    \vskip 1em
    {\songti\zihao{4} \@author \par}
    \vskip 0.5em
    {\zihao{4} \@date \par}
  \end{center}
}
\makeatother

% =====================================================
% 文档信息(请根据需要修改)
% =====================================================
\title{\heiti\zihao{2} 文档标题}
\author{\songti 作者姓名}
\date{\today}

% =====================================================
% 文档开始
% =====================================================
\begin{document}

% =====================================================
% 封面页
% =====================================================
\begin{titlepage}
    \centering
    \vspace*{4cm}
    {\heiti\zihao{2} 文档标题\par}
    \vspace{2cm}
    {\songti\zihao{4} 作者姓名\par}
    \vspace{0.5cm}
    {\zihao{4} \today\par}
    \vfill
\end{titlepage}

\clearpage

% =====================================================
% 目录页(可选)
% =====================================================
% \tableofcontents
% \clearpage

% =====================================================
% 正文开始
% =====================================================
\section{概述}
这是一个标准化的LaTeX文档模板,适用于中文技术文档的编写。

\subsection{模板特性}
本模板包含以下特性:
\begin{itemize}
    \item 完整的中文支持
    \item 美化的章节标题
    \item 优化的表格格式
    \item 超链接支持
    \item 图片插入功能
    \item 参考文献支持
\end{itemize}

\section{使用说明}
\subsection{基本使用}
1. 修改文档信息部分的标题、作者和日期
2. 在正文部分编写内容
3. 如需图片,请将图片放置在Figures文件夹中
4. 如需参考文献,请创建相应的.bib文件

\subsection{表格示例}
\begin{table}[htbp]
    \centering
    \begin{tabular}{|c|l|c|}
        \hline
        序号 & 项目名称 & 状态 \\
        \hline
        1 & 项目一 & 完成 \\
        2 & 项目二 & 进行中 \\
        3 & 项目三 & 计划中 \\
        \hline
    \end{tabular}
    \caption{项目进度表}
    \label{tab:progress}
\end{table}

\subsection{图片示例}
% \begin{figure}[htbp]
%     \centering
%     \includegraphics[scale=0.8]{example.pdf}
%     \caption{示例图片}
%     \label{fig:example}
% \end{figure}

\section{总结}
本模板提供了完整的中文技术文档编写框架,可根据具体需求进行定制。

% =====================================================
% 参考文献(可选)
% =====================================================
% \bibliography{references}

\end{document}