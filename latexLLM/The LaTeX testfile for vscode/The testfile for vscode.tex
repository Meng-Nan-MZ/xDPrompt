\documentclass[fontset=windows]{article}
\usepackage[margin=1in]{geometry}
\usepackage{ctex}
% \usepackage{times}
% \usepackage{newtxtext,newtxmath}
\usepackage{float}
\usepackage{setspace}
\usepackage{lipsum}
\usepackage{graphicx}
\graphicspath{{Figures/}}

\usepackage{enumitem}
\usepackage{hyperref}
\hypersetup{
    colorlinks = true,
    linkcolor = blue,
    urlcolor = blue,
    citecolor = blue,
    bookmarksopen = true,
    bookmarksnumbered = true,
    pdftitle = {This is a testfile for vscode},
    pdfauthor = {Ali-loner}
}
\bibliographystyle{plain}
\newcommand{\upcite}[1]{\textsuperscript{\cite{#1}}}

\renewcommand{\contentsname}{\centerline{Contents}}

% 美化:页眉页脚
\usepackage{fancyhdr}
\pagestyle{fancy}
\fancyhf{}
\fancyhead[L]{1553B IP核验证方案}
\fancyhead[R]{\leftmark}
\fancyfoot[C]{\thepage}

% 美化:章节标题
\usepackage{titlesec}
\titleformat{\section}
  {\normalfont\Large\bfseries\color{blue}}{\thesection}{1em}{}
\titleformat{\subsection}
  {\normalfont\large\bfseries\color{black}}{\thesubsection}{1em}{}
\titleformat{\subsubsection}
  {\normalfont\normalsize\bfseries\color{black}}{\thesubsubsection}{1em}{}

% 美化:表格
\usepackage{booktabs}
\usepackage{array}
\renewcommand{\arraystretch}{1.2}
\setlength{\tabcolsep}{8pt}

% 美化:项目符号
\renewcommand{\labelitemi}{\textbullet}

% 美化:超链接颜色
\usepackage{xcolor}
\definecolor{myblue}{RGB}{0,76,153}
\hypersetup{linkcolor=myblue, urlcolor=myblue, citecolor=myblue}

% 美化:段落间距
\setlength{\parskip}{0.5em}
\setlength{\parindent}{2em}

% 美化:标题居中
\makeatletter
\renewcommand{\maketitle}{
  \begin{center}
    {\heiti\zihao{2} \@title \par}
    \vskip 1em
    % \@author
    % \vskip 1em
    % \@date
  \end{center}
}
\makeatother

% =====================================================
% 目录美化
% =====================================================
\usepackage{tocloft}
\renewcommand{\cftsecleader}{\cftdotfill{\cftdotsep}}
\setlength{\cftbeforesecskip}{2pt}
\setlength{\cftbeforesubsecskip}{1pt}

% =====================================================
% 文档信息
% =====================================================
\title{\heiti\zihao{2} 基于通用星载计算机1553B总线IP核的故障诊断及容错机制升级项目实施方案}
\author{\songti Ali-loner}
\date{\today}

% =====================================================
% 文档开始
% =====================================================
\begin{document}

% =====================================================
% 封面页
% =====================================================
\begin{titlepage}
    \centering
    \vspace*{4cm}
    {\heiti\zihao{2} 基于通用星载计算机1553B总线IP核的故障诊断及容错机制升级项目实施方案\par}
    \vspace{2cm}
    {\songti\zihao{4} 西安电子科技大学ICTT实验室\par}
    \vspace{0.5cm}
    {\zihao{4} \today\par}
    \vfill
\end{titlepage}

\clearpage

% 恢复页眉页脚
\fancypagestyle{main}{
  \fancyhf{}
  \fancyhead[L]{1553B IP核验证方案}
  \fancyhead[R]{\leftmark}
  \fancyfoot[C]{\thepage}
}
\pagestyle{main}

% =====================================================
% 目录页
% =====================================================
\tableofcontents
\clearpage

% =====================================================
% 正文开始
% =====================================================

\section{概述}
1553B总线IP核是星载电子系统中重要的通信接口,负责实现控制器(BC)、远程终端(RT)和总线监控器(BM)之间的数据传输。采用IP核实现1553B控制器功能,不仅能够实现电子系统的小型化设计,而且对于降低功耗、降低成本同样具有重要的意义。与此同时,1553B IP核的可靠性之间影响系统的可靠性,因此提高1553B IP核的可靠性具有重要意义。


\section{引用文档}
《基于通用星载计算机1553B总线IP核的故障诊断及容错机制升级任务书》

\section{1553B总线IP核及升级要求}
\subsection{IP核概述}
1553B总线IP核集成了BC、RT、MT三种控制器,同时寄存器、
存储器的访问方式兼容61580芯片,其整体架构如图\ref{1}所示:

\begin{figure}[htbp]
	\centering
	\includegraphics[scale=0.5]{图1.pdf}
	\caption{IP核结构框图}
	\label{1}
\end{figure}

\subsection{IP核的功能特性}

1553B总线IP核具备以下主要特性:

\begin{itemize}
    \item 集成BC、RT和MT三种工作模式,支持通过软件灵活配置;
    \item 支持1553B总线协议规定的全部10种消息类型;
    \item 支持A、B双冗余通道,并可实现自主切换;
    \item 兼容61580芯片的寄存器访问方式及存储器管理机制;
    \item 提供标准双口RAM,支持按用户需求裁剪容量;
    \item \textbf{BC(总线控制器)特性:}
    \begin{itemize}
        \item 可配置的A/B工作区;
        \item 多种可编程中断类型;
        \item 自动重发功能,支持可编程的重发次数及通道选择;
        \item 可编程的帧自动重复发送功能;
        \item 可编程的消息间隔时间;
    \end{itemize}
    \item \textbf{RT(远程终端)特性:}
    \begin{itemize}
        \item 可配置的A/B工作区;
        \item 多种可编程中断类型;
        \item 可编程的RT地址;
        \item 支持单缓冲、循环缓冲及双缓冲等多种存储器管理方案;
        \item 可编程的非法命令表;
        \item 可编程的MODE代码中断表;
        \item 可编程的子地址忙表;
    \end{itemize}
    \item \textbf{MT(总线监控器)特性:}
    \begin{itemize}
        \item 支持字监视模式;
        \item 支持可选的消息监视模式;
    \end{itemize}
\end{itemize}

\subsection{升级要求}

针对本项目的升级任务,需重点开展以下工作:

\begin{itemize}
    \item \textbf{已知缺陷修复:}
    \begin{enumerate}[label=(\arabic*)]
        \item 误触发中断:在特定工作场景下,存在重复触发中断的问题;
        \item 通道切换故障:在甲方研制的RV芯片上进行1553B总线通道切换测试时,无法按照IP核手册要求实现正常切换;
        \item RAM堆栈空间被改写:在某些场景下,1553B IP核存储于RAM空间的状态信息及消息结果被异常覆盖;
        \item RT中断状态无法清除:RT工作于循环缓冲模式时,产生的中断状态无法被正常清除;
        \item 国产1553B收发器拖尾问题导致接收异常:使用国产收发器时,出现消息接收异常现象。
    \end{enumerate}
    \item \textbf{功能完善:}
    \begin{enumerate}[label=(\arabic*)]
        \item MT模式下,字监测功能需支持存储空间配置;
        \item 1553B IP核的RAM空间由固定4K$\times$16bit,升级为4K$\times$16~64K$\times$16可配置;
        \item 开发专用的1553B IP核验证平台,提升验证效率和覆盖面。
    \end{enumerate}
    \item \textbf{质量要求:}
    \begin{enumerate}[label=(\arabic*)]
        \item 关键故障修复需达到预期效果,确保系统稳定可靠运行;
        \item 验证平台搭建需稳定可靠,能够支持多种故障场景的模拟与测试;
        \item 测试用例需充分覆盖全部功能点,确保高代码覆盖率;
        \item SOC集成验证需确保IP核与甲方SOC原型系统的兼容性与稳定性。
    \end{enumerate}
\end{itemize}


\begin{figure}[htbp]
	\centering
	\includegraphics[scale=0.8]{Figures/图2.pdf}
	\caption{升级工作流程图}
	\label{fig:platform}
\end{figure}


\section{升级策略}
\subsection{升级流程}
本项目的输入包括任务书、IP核数据手册等相关文档,以及存放于甲方服务器上的RTL代码。
整体升级流程分为三步,如图\ref{fig:platform}所示:



\begin{enumerate}[label=(\arabic*)]
    \item \textbf{功能点梳理与验证平台开发}:梳理1553B IP核的全部功能点,开发专用的验证平台,为后续测试和验证工作提供基础环境。
    \item \textbf{测试用例开发与功能验证}:针对各功能点,尤其是与已知缺陷相关的部分,设计并实现测试用例,系统性验证功能正确性。采用故障注入技术,重点测试IP核的故障诊断与容错能力。
    \item \textbf{缺陷处理与升级复测}:对发现的缺陷进行详细记录和分析,制定合理的升级方案,完成升级后进行复测,确保问题彻底解决。整个过程与甲方保持数据同步,确保信息一致。
\end{enumerate}

\subsection{验证}

针对1553B总线的网络拓扑接口,本项目设计的验证平台如图所示。
1553B总线是一种主从式半双工通信总线,网络中包含1个总线控制器(BC)、最多31个远程终端(RT)和1个总线监控器(MT),并采用双冗余总线结构。
验证平台说明如图\ref{fig:structure}所示:

\begin{figure}[htbp]
	\centering
	\includegraphics[scale=0.6]{Figures/图3.pdf}
	\caption{验证平台结构示意图}
	\label{fig:structure}
\end{figure}


\begin{enumerate}[label=(\arabic*)]
    \item BC、RT0-30、MT均为1553B IP核实例,分别配置为不同的功能模式。其中RT0~RT30的编号对应1553B协议规定的31个RT地址。
    \item APB MASTER为APB总线主设备,直接连接各IP核,负责初始化、消息配置及结果读取。
    \item 1553B收发器用于实现总线信号的三态驱动和信号畸变等功能,模拟真实物理总线环境。
    \item 测试用例集针对1553B IP核各功能点设计,所有消息均由BC发起,APB master负责读取和配置,消息在总线上传输。
    \item 各模式(BC、RT、MT)会将测试结果写入log文件,便于按RT地址检索和分析。
\end{enumerate}

验证平台在Linux环境下搭建,采用VCS201809作为仿真工具。平台目录结构如下:DUT位于rtl目录,仿真模型在tb目录,脚本存放于vcs目录,验证激励放在testcase目录。通过vcs目录下的Makefile运行仿真,最终生成波形文件my.vpd和log信息文件。

\section{功能点及验证用例设计}
依据IP核的实际工作场景,该IP核中有1553B编解码器的时钟,固定为16MHz,APB时钟,与16Mhz时钟为异步时钟,并且甲方测试发现,apb时钟在不同频率下,该IP核表现出了不同的结果,因此本次测试用的apb时钟分别设置为16MHz,100Mhz及500Mhz。
\subsection{寄存器测试}
1553B IP核的寄存器列表及属性如表\ref{tab:register-list}所示,使用APB MASTER进行读写测试。

\begin{table}[htbp]
    \centering
    \begin{tabular}{|c|l|c|}
        \hline
        地址 & 寄存器名称 & 访问方式 \\
        \hline
        0   & 中断屏蔽寄存器             & RD/WR \\
        1   & 配置寄存器1                & RD/WR \\
        2   & 配置寄存器2                & RD/WR \\
        3   & 启动/复位寄存器            & WR    \\
            & 命令堆栈指针寄存器         & RD    \\
        4   & BC控制字/RT子地址控制字寄存器 & RD \\
        5   & 时间戳寄存器               & RD    \\
        6   & 中断状态寄存器             & RD    \\
        7   & 配置寄存器3                & RD/WR \\
        8   & 配置寄存器4                & RD/WR \\
        9   & 配置寄存器5                & RD/WR \\
        A   & 数据堆栈寄存器             & RD    \\
        B   & BC帧时间保持寄存器         & RD    \\
        C   & BC下一条消息时间保持寄存器 & RD    \\
        D   & RT上一个命令字寄存器       & RD    \\
        E   & RT状态字寄存器             & RD    \\
        F   & RT BIT字寄存器             & RD    \\
        \hline
    \end{tabular}
    \caption{1553B IP核寄存器列表及访问方式}
    \label{tab:register-list}
\end{table}

\subsection{RAM测试}
1553B IP核有私有4K*16bit的RAM,进行读写测试。

\subsection{1553B收发器拖尾测试}
故障注入,模拟1553B收发器拖尾的现象,验证引发的故障。



\subsection{BC功能点}

\begin{table}[H]
    \centering
    \begin{tabular}{|c|p{11cm}|}
        \hline
        序号 & 测试内容 \\
        \hline
        1  & 针对1553B协议规定的10种消息类型,分别构造相应的消息,在A、B两条总线上进行发送测试,验证消息类型支持及通道切换功能。 \\
        2  & 测试堆栈区A的消息发送功能,覆盖1条消息至64条消息的不同场景,验证堆栈区A的消息缓存与发送能力。 \\
        3  & 测试堆栈区B的消息发送功能,覆盖1条消息至64条消息的不同场景,验证堆栈区B的消息缓存与发送能力。 \\
        4  & 验证BC模式下消息的发送与存储功能,确保消息能够正确下发并在RAM中正确存储。 \\
        5  & 验证BC模式下消息发送的668us超时功能,确保超时机制能够按协议要求正确响应。 \\
        6  & 配置不同的消息间隔时间,测试BC的消息间隔功能,验证消息发送间隔的可编程性和准确性。 \\
        7  & BC中断测试:分别测试EOM(消息结束)、EOF(帧结束)、特定消息的中断等,验证中断响应的正确性。 \\
        8  & 消息重试功能测试:通过故障注入,发送消息后重试1次,更换总线,重试1次后成功,验证重试及总线切换机制。 \\
        9  & 消息重试功能测试:通过故障注入,发送消息后重试1次,更换总线,重试1次后不成功,验证失败处理机制。 \\
        10 & 消息重试功能测试:通过故障注入,发送消息后重试1次,不更换总线,重试1次后成功,验证同一总线重试机制。 \\
        11 & 消息重试功能测试:通过故障注入,发送消息后重试1次,更换总线,重试1次后不成功,验证多次失败处理。 \\
        12 & 消息重试功能测试:通过故障注入,发送消息后重试1次,更换总线,重试2次后成功,验证多次重试及切换机制。 \\
        \hline
    \end{tabular}
    \caption{BC功能点测试项汇总}
    \label{tab:bc-feature-summary}
\end{table}



\subsection{RT功能点}

\begin{table}[H] % 使用 [H] 强制表格紧跟在当前位置,需要 float 宏包
    \centering
    \begin{tabular}{|c|p{10cm}|}
        \hline
        序号 & 测试内容 \\
        \hline
        1  & RT中断测试,EOM中断 \\
        2  & RT中断测试,SA EOM中断 \\
        3  & RT中断测试,循环缓冲卷回中断,设置缓冲区256、512、1024、2048大小 \\
        4  & RT响应时间测试 \\
        5  & RT响应超时测试-故障注入 \\
        6  & RT发送消息超时测试-故障注入 \\
        7  & RT接收缓冲区配置不同地址的测试 \\
        8  & RT发送缓冲区配置不同地址的测试 \\
        9  & RT广播缓冲区配置不同地址的测试 \\
        10 & RT堆栈A,消息索引功能,测试 \\
        11 & RT堆栈B,消息索引功能,测试 \\
        12 & RT堆栈A,卷回测试 \\
        13 & RT堆栈B,卷回测试 \\
        14 & RT消息存储功能 \\
        15 & RT消息循环缓冲模式-发送消息测试,设置缓冲区256、512、1024、2048大小 \\
        16 & RT消息循环缓冲模式-接收消息测试,设置缓冲区256、512、1024、2048大小 \\
        17 & RT消息双缓冲模式-发送消息测试 \\
        18 & RT消息双缓冲模式-接收消息测试 \\
        19 & RT消息非法化测试 \\
        20 & RT消息子地址忙测试 \\
        \hline
    \end{tabular}
    \caption{RT功能点测试项汇总}
    \label{tab:rt-feature-summary}
\end{table}


\subsection{MT功能点}

\begin{table}[H]
    \centering
    \begin{tabular}{|c|p{10cm}|}
        \hline
        序号 & 测试内容 \\
        \hline
        1 & MT的中断测试 \\
        2 & MT的消息存储测试 \\
        3 & MT的字监视功能测试 \\
        4 & MT的消息监视功能测试 \\
        \hline
    \end{tabular}
    \caption{MT功能点测试项汇总}
    \label{tab:mt-feature-summary}
\end{table}

\section{SOC集成与验证}
在验证过程中,将IP核集成到甲方的SOC中,并进行板级测试。
具体流程包括:首先将经过仿真验证的1553B IP核集成到SOC设计中,完成RTL级的集成和接口适配。随后,进行SOC的综合、布局布线和流片,获得实际的芯片样品。将芯片焊接到开发板或测试板上,搭建完整的硬件测试环境。通过板级测试,验证IP核在SOC系统中的功能、性能和稳定性,包括与其他模块的协同工作、接口时序、数据传输正确性等。同时,结合自动化测试脚本,对各项功能点和异常场景进行全面测试,确保IP核在实际应用环境下的可靠性和兼容性。测试过程中发现的问题将及时反馈并修正,直至满足项目要求。

\section{验证进度安排}

\begin{table}[htbp]
    \centering
    \begin{tabular}{|c|c|c|c|}
        \hline
        序号 & 任务项 & 完成时间 & 交付物 \\
        \hline
        1 & 故障诊断及升级 & 合同签订后6周内 & 设计报告、RTL代码 \\
        \hline
        2 & 仿真验证阶段 & 合同签订后15周内 & 测试报告、验证平台及测试用例 \\
        \hline
        3 & 验收阶段 & 合同签订后16周内 & 研制总结报告 \\
        \hline
    \end{tabular}
    \caption{研制进度安排}
    \label{tab:schedule}
\end{table}

\section{质量保证措施}

针对本项目的验证工作,拟采取以下质量保证措施:

\begin{enumerate}[label=(\arabic*)]
    \item \textbf{基于覆盖率驱动的验证用例设计} \\
    采用覆盖率驱动的验证方法,能够量化验证进度,确保验证的全面性和科学性。本项目设定的覆盖率目标为:功能覆盖率达到100\%,代码测试覆盖率达到90\%(包括分支覆盖率、状态机覆盖率、行覆盖率),对于未覆盖的语句需进行说明。实现功能点100\%覆盖的前提是功能点明确定义,第5章已汇总本设计的全部功能点。在验证实施过程中,将结合代码覆盖率、总线协议符合性等信息,与设计人员协作,持续完善和补充验证点列表。
    \item \textbf{加强阶段性评审} \\
    通过阶段性评审,完善验证场景,补充功能点和验证用例。项目设置至少三个评审节点:1)实施方案评审;2)RTL升级后评审;3)仿真验证结果评审。根据每次评审结果,动态补充和完善验证用例,确保验证工作的系统性和完整性。
\end{enumerate}

\section{风险分析和解决措施}

本项目验证工作主要存在以下风险及应对措施:

\begin{enumerate}[label=(\arabic*)]
    \item \textbf{验证点遗漏风险} \\
    \textbf{风险描述:} 功能点主要依据设计中各模块的主要功能特性进行定义,实际操作中可能存在部分验证点遗漏的情况。\\
    \textbf{解决措施:}(1)采用“功能点+代码覆盖率”相结合的方式,确保验证的全面性。如果功能点均已覆盖但代码覆盖率仍不高,需警惕功能点列举是否存在遗漏;(2)通过阶段性评审或与设计人员协作,持续完善和补充验证点列表,降低遗漏风险。
    \item \textbf{验证进度失控风险} \\
    \textbf{风险描述:} 验证进度失控的主要原因包括:对验证结果缺乏信心,导致不断补充和修改验证用例,影响整体进度。\\
    \textbf{解决措施:} 与设计人员共同梳理和讨论典型工作场景,科学合理地构建测试用例,避免盲目开发。结合功能覆盖率和代码覆盖率等量化指标,客观评价验证用例的质量和有效性,确保验证工作有序推进。
\end{enumerate}

\bibliography{books}
\end{document}